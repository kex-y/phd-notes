\documentclass[]{article}
\usepackage{amsmath}
\usepackage[utf8]{inputenc}
\usepackage[T1]{fontenc}
\usepackage[bitstream-charter]{mathdesign}
\usepackage[dvipsnames]{xcolor}
\usepackage{hyperref}
\usepackage{xcolor}
\usepackage[margin = 1.5in]{geometry}
\usepackage{graphicx}
\usepackage{tikz}
\usepackage{tikz-cd}
\usepackage{physics}
\usepackage{amsthm}
\usepackage{titling}
\usepackage{mathtools}
\usepackage{authblk}
\usepackage{todonotes}
\setlength{\parindent}{0pt}
\theoremstyle{definition}
\newtheorem{theorem}{Theorem}
\newtheorem{assumption}{Assumption}
\newtheorem*{theorem*}{Theorem}
\newtheorem{prop}{Proposition}
\newtheorem{corollary}{Corollary}[theorem]
\newtheorem*{remark}{Remark}
\theoremstyle{definition}
\newtheorem{definition}{Definition}
\newtheorem{lemma}{Lemma}
\newtheorem*{lemma*}{Lemma}
\newtheorem{proposition}{Proposition}
\newtheorem*{proposition*}{Proposition}
\newtheorem{example}{Example}[section]

\let\phi\varphi
\let\theta\vartheta
\let\epsilon\varepsilon
\let\emptyset\varnothing
\def\pvar{p\mathrm{\text{-}var}}

\setlength{\parskip}{3pt}

\begin{document}
\title{}
\author{}
\maketitle

\section*{\(p\)-Variation vs \(\alpha\)-H\"older Continuity}

For \(\alpha \in (0, 1]\) and \(\gamma \in \mathcal{C}^\alpha([0, T], V)\), it is well-known that 
\(\gamma \in \mathcal{C}^{\pvar}([0, T], V)\) for \(p = \alpha^{-1}\) and 
\[\|\gamma\|_{\pvar} \le T^\alpha \|\gamma\|_\alpha.\]
Moreover, if \(\gamma\) is a continuous path of finite \(p\)-variation for some \(p > 1\), we 
can reparametrize \(\gamma\) (so that its \(p\)-variation norm grows with constant speed) to 
obtain a \(p^{-1}\)-H\"older path. In this case, one can make sure that the reparametrized process 
has unit H\"older norm. 

A question arises on whether or not, starting from a H\"older path, we can bound its corresponding 
\(p\)-variation norm from below by its H\"older norm. The answer is in general negative. Indeed, 
philosophically, the H\"older norm is a local property whilst the \(p\)-variation norm is global. 
Thus, the H\"older norm cannot capture the global behavior of the path and consequently the two 
norms cannot be equivalent in general. To be more precise, we know that the \(p\)-variation is 
invariant under reparametrization. Thus, if there exists non-decreasing non-negative function \(f_T\) for which 
\[f_T(\|\gamma\|_{\alpha}) \le \|\gamma\|_{\pvar},\]
then, reparametrizing \(\gamma\) by increasing the speed on the interval for which it H\"older norm 
attains its maximum, and compensating by decreasing the speed on the rest of the interval, the 
left hand side can be made arbitrarily large whilst keeping the right hand side constant. 

We remark that the above argument cannot be applied to the upper bound. More precisely, we cannot 
reparametrize a path \(\gamma\) to make its H\"older norm arbitrarily small whilst keeping its 
domain invariant. 

\section*{Interpolation}

By viewing \(L^\infty\) as \(C^0\) (here \(C^0\) is not the space of continuous paths but the 
"H\"older space" with 0 H\"older exponents), we expect a interpolation inequality 
\[L^0 \to L^p \to L^\infty = C^0 \to C^\alpha \to C^k \to C^\infty.\]
An example of this is from \cite[Lemma A.3]{Hairer:10}:
\[\|f\|_\infty \le 2 \left(T^{-\frac{1}{2}} \|f\|_{L^2} \vee 
  \|f\|_{L^2}^{\frac{2\alpha}{2\alpha + 1}} \|f\|_{\alpha}^{\frac{1}{2\alpha + 1}}\right)\]
for any \(f \in \mathcal{C}^\alpha([0, T])\).

\section*{\(\theta\)-H\"older Roughness and Control of the Gubinelli Derivative}

We have a natural understanding of the H\"older roughness of a path. In particular, although we 
have the embeddings between H\"older spaces, we do not usually consider a constant path to be 
a \(\frac{1}{2}\)-H\"older path, nor a Brownian motion to be a \(\frac{1}{3}\)-H\"older path.
Namely, we would like to work on the boundary of how rough the path is. In the context of rough 
path theory, working on the boundary guarantees uniqueness of the Gubinelli derivative. 

For simplicity, let us consider the one-dimensional case: Let 
\((X, \mathbb{X}) \in \mathcal{C}^\alpha([0, T])\) be such that, for some \(s \in [0, T]\), there 
\[\limsup_{t \to s} \frac{|X_{st}|}{|t - s|^{\theta}} = \infty\]
for some \(\theta \in (\alpha, 2 \alpha)\). Then, if \((Y, Y') \in \mathcal{D}^\alpha_X\), we have 
that 
\[Y'_s = \frac{Y_{st}}{X_{st}} + \frac{R_{st}}{|t - s|^{2\alpha}} \frac{|t - s|^\theta}{X_{st}} |t - s|^{2\alpha - \theta}.\]
Thus, taking \(t \to s\), we have that \(Y'_s = \liminf_{t \to s} \frac{Y_{st}}{X_{st}} =: \partial_X Y_s\).
Namely, \(Y'_s\) is uniquely determined by \(X\) and \(Y\).

In general, for a rough path \(\mathbf{X} = (X, \mathbb{X}) \in \mathcal{C}^\alpha([0, T])\) which is \(\theta\)-H\"older rough, 
denoting \(L_\theta(X)\) for its modulus of \(\theta\)-H\"older roughness, we have the estimate \cite[Proposition 1]{Hairer:10}
\[L_{\theta}(X)\|Y'\|_{\infty} \lesssim \|Y\|_{\infty}\epsilon^{-\theta} + \|R^Y\|_{2\alpha} \epsilon^{2\alpha - \theta}\]
for any \((Y, Y') \in \mathcal{D}^\alpha_X\) and \(\epsilon \in (0, \frac{T}{2}]\). Thus, in the case 
where \((Y, Y') \in \mathcal{D}^\alpha_X\) is \((\mathbf{X}, \phi)\)-coherent, 
\(\phi(Y), \partial_X \phi(Y), \partial_X^2 \phi(Y) \dots\) 
can all be controlled by \(Y\) and \(R^Y\):
\begin{align*}
  \|\phi(Y)\|_{\infty} & \lesssim 
    (L_{\theta}(X))^{-1}(\|Y\|_{\infty}\epsilon_0^{-\theta} + \|R^Y\|_{2\alpha} \epsilon_0^{2\alpha - \theta})\\
  \|\partial_X \phi(Y)\|_{\infty} & \lesssim 
    (L_{\theta}(X))^{-1}(\|\phi(Y)\|_{\infty}\epsilon_1^{-\theta} + \|R^Y\|_{2\alpha} \epsilon_1^{2\alpha - \theta})\\
    & \lesssim (L_\theta(X))^{-2} \|Y\|_\infty \epsilon_0^{-\theta} \epsilon_1^{-\theta} + 
       (L_\theta(X))^{-1} \|R^Y\|_{2\alpha} (((L_\theta(X)))^{-1} \epsilon_1^{-\theta} \epsilon_0^{2\alpha - \theta} 
        + \epsilon_1^{2\alpha - \theta})\\
  & \cdots
\end{align*}
for small \(\epsilon_0, \epsilon_1, \dots\). We can even bound the H\"older norm of \(\partial_X \phi(Y)\) since 
\[\|\partial_X \phi(Y)\|_\alpha \le \|R^{\phi(Y)}\|_{2\alpha} T^\alpha + \|X\|_\alpha \|\partial_X^2 \phi(Y)\|_\infty.\]
Nonetheless, this control is not sufficient for applications in \cite[Section 4.4]{Li:25} where 
\(\|Y\|_{\infty; I} < R(k + 1)\) for some known constants \(R, k\) no matter what choice of \(\epsilon\) 
we choose. In particular, for this application, we find that we will need to take \(\epsilon\) large 
in order to control the term \(\|Y\|_\infty \epsilon^{-\theta}\). This motivates a long range version 
of \(\theta\)-H\"older roughness: We say \(X\) has \(\theta\)-long range variation if, 
for every \(s \in \mathbb{R}_+\) and \(r > 1\)
\[\sup_{|t - s| \le r} |X_{st}| \ge U_\theta(X) r^\theta.\]
By the same arguments as in \cite{Hairer:13}, the same bound holds assuming long range variation 
but now we allow for large \(\epsilon > 1\).

It is not difficult to see that a fractional Brownian motion \(B^H\) with Hurst parameter \(H\) satisfies 
the above definition for any \(\theta < H\). This is in contrast to \(\theta\)-H\"older roughness 
in which \(B^H\) is \(\theta\)-H\"older rough for any \(\theta > H\). This is unfortunate in the 
sense that, for applications in \cite{Li:25}, we need \(\theta\)-long range variation for 
\(\theta > 1 - \alpha\) with \(\alpha\) being the H\"older exponent of the driving rough path which 
is not possible. Indeed, philosophically, we can already see that these controls cannot improve 
the result as the assumptions say nothing about \(\phi\).



\bibliographystyle{alpha}
\bibliography{refs}

\end{document}
