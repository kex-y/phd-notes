\documentclass[]{article}
\usepackage{amssymb,amsmath}
\usepackage[utf8]{inputenc}
\usepackage[bitstream-charter]{mathdesign}
\usepackage[T1]{fontenc}
\usepackage[dvipsnames]{xcolor}
\usepackage{hyperref}
\hypersetup{
  pdftitle={Notes on the Cameron-Martin Space},
  pdfauthor={Kexing Ying},
  colorlinks=true,
  linkcolor=Maroon,
  filecolor=Maroon,
  citecolor=Maroon,
  urlcolor=Maroon}
\usepackage{xcolor}
\usepackage[margin = 1.5in]{geometry}
\usepackage{graphicx}
\usepackage{physics}
\usepackage{amsthm}
\usepackage{mathtools}

\setlength{\parindent}{0pt}
\theoremstyle{definition}
\newtheorem{theorem}{Theorem}
\newtheorem*{theorem*}{Theorem}
\newtheorem*{definition*}{Definition}
\newtheorem{prop}{Proposition}
\newtheorem{corollary}{Corollary}[theorem]
\newtheorem*{remark}{Remark}
\theoremstyle{definition}
\newtheorem{definition}{Definition}[section]
\newtheorem{lemma}{Lemma}[section]
\newtheorem{proposition}{Proposition}[section]
\newtheorem*{proposition*}{Proposition}
\newtheorem{example}{Example}[section]

\setlength{\parskip}{3pt}

\title{Notes on the Cameron-Martin Space}
\author{Kexing Ying}

\begin{document}

\maketitle

Notes taken during the \textit{Introduction to SPDEs} lectures given by Martin Hairer 
(2024 Spring at EPFL).

\section*{Abstract theory}

We look at Gaussian measures on (mostly infinitely dimensional) Banach spaces. While on \(\mathbb{R}^d\), 
the notion of a Gaussian measure can be easily defined by specifying its density with respect to the 
Lebesgue measure, this is clearly not case for infinite dimensional spaces. Instead, since for \(X\) a 
separable Banach space, any Borel measure \(\mu\) on \(X\) is uniquely determined by its projections 
on its linear functionals, we define a Gaussian measure on \(X\) by requiring that its projections are 
real Gaussians. 

We will take \(X\) to be a separable Banach space and \(H\) a Hilbert space from this point forward.
We denote \(\mathcal{M}(X)\) and \(\mathcal{M}(H)\) to be the space of Borel probability measures 
on \(X\) and \(H\) respectively.

\begin{lemma}
  For \(\mu, \nu \in \mathcal{M}(X)\), \(\mu = \nu\) if and only if for all \(l \in X^*\), we have that 
  \(l_* \mu = l_* \nu\).
\end{lemma}

\begin{definition}
  \(\mu \in \mathcal{M}(X)\) is a Gaussian measure if for all \(l \in X^*\), denoting \(l_* \mu\) the 
  push-forward measure of \(\mu\) along \(l\), we have that \(l_* \mu\) is a Gaussian measure on \(\mathbb{R}\).
  We say \(\mu\) is centered if for all \(l \in X^*\), \(l_* \mu\) is centered.
\end{definition}

We focus on the case where \(\mu\) is centered as we can always translate \(\mu\) if needed.

\begin{definition}
  For \(\mu \in \mathcal{M}(X)\) a Gaussian measure, we define its covariance by 
  \[C_\mu : X^* \times X^* \to \mathbb{R} : (l, l') \mapsto \int_X l(x) l'(x) \mu(\dd x).\] 
  By the property of the integral, it is clear that the covariance operator is a symmetry, positive semi-definite 
  bilinear form.
\end{definition}

\begin{definition}
  For \(\mu \in \mathcal{M}(X)\) a Gaussian measure, we define its Fourier transform by 
  \[\hat \mu(l) = \int_X e^{il(x)} \mu(\dd x) = \exp \left(-\frac{1}{2} C_\mu(l, l)\right)\]
  where the second equality is due to properties of the real Gaussian. Similar to the finite dimensional 
  case, \(\mu\) is uniquely determined by its Fourier transform (this is a general property not just for Gaussians).
\end{definition}

The covariance of \(\mu\) can be viewed in many ways most notably, 
\[C_\mu : X^* \to X : l \mapsto \int_X x l(x) \mu(\dd x)\]
for which the integral is well-defined (by Fernique, see below) as a Bochner integral. 
We then have \(C_\mu(l, l') = l'(C_\mu l)\).

An important theorem for the Gaussian measure is Fernique's theorem which shows that the Gaussian has 
finite moments of all orders (via exponential Chebyshev).

\begin{theorem}[Fernique's Theorem]
  For \(\mu \in \mathcal{M}(X)\) a Gaussian measure, there exists a constant \(\alpha > 0\) such that 
  \[\int_X \exp(\alpha \norm{x}^2) \mu(\dd x) < \infty.\]
\end{theorem}

As a consequence of Fernique's, any Gaussian measure has finite variance and thus, the covariance operator
is bounded. 

\begin{proposition}
  For \(\mu \in \mathcal{M}(H)\) a Gaussian measure, viewing \(C_\mu : H \to H\) applying Riesz representation 
  whenever needed, we have 
  \[\int_H \|x\|^2 \mu(\dd x) = \tr C_\mu = \sum_{k = 1}^\infty \langle C_\mu(e_k), e_k\rangle\]
  where \((e_k)_{k = 1}^\infty\) is an orthonormal basis of \(H\). Namely, \(C_\mu\) is 
  positive, self-adjoint and trace class. Conversely, if \(C\) is a positive, self-adjoint and trace class 
  operator, then there exists a Gaussian measure \(\mu\) on \(H\) such that \(C = C_\mu\).
\end{proposition}

While so far, the definition and the behavior of the Gaussian on Banach space is seemingly intuitive, 
this cannot be further from the truth should we look a bit closer. While, in the finite dimensional case,
the Gaussian measures are always equivalent to the Lebesgue measure, and consequently each other, 
in infinite dimensions, Gaussians a easily mutually singular. As we shall see, in fact with probability 
one, any translation of the Gaussian will be mutually singular to the original Gaussian. The 
Cameron-Martin space is then the linear subspace of \(X\) with zero measure for which translation 
dose not make the Gaussian measure singular. While it has measure zero, it turns out that the 
Cameron-Martin space characterized the Gaussian measure completely.

\begin{definition}
  For \(\mu \in \mathcal{M}(X)\) a Gaussian measure, we define 
  \[\mathring H_\mu = \{h \in X : \exists h^* \in X^*, C_\mu(h^*) = h\} = C_\mu(X^*),\]
  where we equip it with the inner product \(\langle h, k\rangle_\mu = C_\mu(h^*, k^*)\). Then, the 
  Cameron-Martin space \(H_\mu\) of \(\mu\) is defined as the closure of \(\mathring H_\mu\) under 
  its norm \(\norm{\cdot}_\mu\).
\end{definition}

Despite the norm under which the Cameron-Martin space is completed over is different, the Cameron-Martin 
space is always a subspace of the original Banach space.

\begin{theorem}[Cameron-Martin]
  Let \(\mu \in \mathcal{M}(X)\) be a Gaussian.
  For \(h \in X\), denoting \(\tau_h : x \in X \mapsto x + h\), we have that \(\tau_h^* \mu \ll \mu\) if and only 
  if \(h \in H_\mu\). In this case, 
  \[\dv{\tau_h^* \mu}{\mu} = \exp(h^*(x) - \frac{1}{2}\|h\|^2_\mu).\]
\end{theorem}

\begin{proposition}
  \[H_\mu = \bigcap_{V \le X; \mu(V) = 1} V,\]
  and \(\mu(H_\mu) = 0\) if and only if \(\dim(X) = \infty\).
\end{proposition}

As a consequence, if \(A \in \mathcal{L}(X)\) a linear isomorphism, denoting \(\nu = A_* \mu\), 
we have that 
\[H_\nu = \bigcap_{V \le X; \nu(V) = 1} V = \bigcap_{V \le X; \mu(A^{-1}V) = 1} V = 
  \bigcap_{V \le X; \mu(V) = 1} AV = A(H_\mu)\]

\section*{Examples}

Let \(\mu\) be the Gaussian measure on \(\mathcal{C}([0, 1], \mathbb{R})\) with covariance 
\(C_\mu(\delta_t, \delta_s) = s \wedge t\). \(\mu\) corresponds to the law of the Brownian motion 
on \([0, 1]\) and we call it the Wiener measure. We have that \(H_\mu = H^{1, 2}_0([0, 1])\) - 
the space of all absolutely continuous functions \(h\) with \(h(0) = 0\) and \(\dot h(s) \in L^2([0, 1])\). Moreover, for \(h \in H_\mu\), 
\(h^*(k) = \int_0^1 \dot h(s) \dot k(s) \dd s\). As this defines a bounded linear functional from \(H_\mu\), 
we can almost surely uniquely extend \(h^*\) to a measurable element of \(C_\mu(\delta_t, \delta_s)^*\). To see this, it suffices to check that 
\[h_t = \delta_t(h) = C_\mu(\delta_t, h^*) = h^*(C_\mu(\delta_t)).\]
We see that 
\[C_\mu(\delta_t)_s = \left(\int \omega \delta_t(\omega) \mu(\dd \omega)\right)_s 
  = \int \omega_s \omega_t \mu(\dd \omega) = C_\mu(\delta_t, \delta_s) = s \wedge t.\]
Thus, \(\dot{C}_\mu(\delta_t)_s = \mathbb{1}_{[0, t]}(s)\) and hence,
\[h^*(C_\mu(\delta_t)) = \int_0^t \dot f_s \dd s = f_t\]
as required.

As an application, the Cameron-Martin theorem allows us to bound the probability that a Brownian motion 
stays near a given function. In particular, for \(f \in H^{1, 2}_0([0, 1])\), denoting \(\mu_W\) for the 
Wiener measure, we have that
\[\mathbb{P}(|B_t - f|_\infty \le \epsilon) = \tau_{-f}^* \mu_W(\overline B_\epsilon(0))
  = e^{-\frac{1}{2}\|f\|^2_{H^{1, 2}_0([0, 1])}} \int_{\overline B_\epsilon(0)} e^{- f^*(\omega)} \mu_W(\dd \omega).\]
This is then bounded above by 
\[e^{-\frac{1}{2}\|f\|^2_{H^{1, 2}_0([0, 1])} + \epsilon \|f^*\|_{X^*}} \mathbb{P}(|B_t|_\infty \le \epsilon).\]
On the other hand, using the inequality \(e^x \ge 1 + x\), we have that 
\[\int_{\overline B_\epsilon(0)} e^{- f^*(\omega)} \mu_W(\dd \omega) \ge 
  \int_{\overline B_\epsilon(0)} (1 + - f^*(\omega)) \mu_W(\dd \omega)
  \ge \mathbb{P}(|B_t|_\infty \le \epsilon) - \int_{\overline B_\epsilon(0)} f^*(\omega) \mu_W(\dd \omega)\]
Now, since \(f^*_* \mu_W \sim \mathcal{N}(0, \|f\|_{H_W}^2)\), the last integral is zero by spherical symmetry. 
Consequently, have the bound 
\[\mathbb{P}(|B_t - f|_\infty \le \epsilon) \in 
  e^{-\frac{1}{2}\|f\|^2_{H^{1, 2}_0([0, 1])}} 
  \left[\mathbb{P}(|B_t|_\infty \le \epsilon), e^{\epsilon \|f^*\|_{X^*}}\mathbb{P}(|B_t|_\infty \le \epsilon)\right]\]
and thus, 
\[\lim_{\epsilon \downarrow 0} \frac{\mathbb{P}(|B_t - f|_\infty \le \epsilon)}{\mathbb{P}(|B_t|_\infty \le \epsilon)} 
  = e^{-\frac{1}{2}\|f\|^2_{H^{1, 2}_0([0, 1])}}.\]

\end{document}