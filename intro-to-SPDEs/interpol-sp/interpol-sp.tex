\documentclass[]{article}
\usepackage{amssymb,amsmath}
\usepackage[utf8]{inputenc}
\usepackage[bitstream-charter]{mathdesign}
\usepackage[T1]{fontenc}
\usepackage[dvipsnames]{xcolor}
\usepackage{hyperref}
\hypersetup{
  pdftitle={Interpolation space},
  pdfauthor={Kexing Ying},
  colorlinks=true,
  linkcolor=Maroon,
  filecolor=Maroon,
  citecolor=Maroon,
  urlcolor=Maroon}
\usepackage{xcolor}
\usepackage[margin = 1.5in]{geometry}
\usepackage{graphicx}
\usepackage{physics}
\usepackage{amsthm}
\usepackage{mathtools}

\setlength{\parindent}{0pt}
\theoremstyle{definition}
\newtheorem{theorem}{Theorem}
\newtheorem*{theorem*}{Theorem}
\newtheorem{prop}{Proposition}
\newtheorem{corollary}{Corollary}[theorem]
\newtheorem*{remark}{Remark}
\theoremstyle{definition}
\newtheorem*{definition}{Definition}
\newtheorem{lemma}{Lemma}[section]
\newtheorem*{proposition}{Proposition}
\newtheorem{example}{Example}[section]

\setlength{\parskip}{3pt}

\title{Interpolation space}
\author{Kexing Ying}

\begin{document}

\maketitle

Notes taken during the \textit{Introduction to SPDEs} lectures given by Martin Hairer 
(2024 Spring at EPFL).

Throughout this note, we take \(L\) the generator of a analytic semigroup \(S\) on the Banach space \(\mathcal{B}\) which satisfies 
\(\|S(t)\| \le M e^{-w t}\) for some \(w > 0\) so that the resolvent of \(L\), \(\rho(L)\) contains 
the right half of the complex plane (recall \(R_\lambda = \int_0^\infty e^{-\lambda t} S(t) \dd{t}\) 
which is well-defined for all \(\lambda \in \mathbb{C}, \text{Re}(\lambda) > 0\)). 

For \(\alpha > 0\), by viewing the formal expression \(S(t) = e^{tL}\), by making an appropriate substitution, we have 
the following computation
\[\int_0^\infty t^{\alpha - 1} e^{tL} \dd t = (-L)^{-\alpha} \int_0^\infty t^{\alpha - 1} e^{-t} \dd t = (-L)^{-\alpha} \Gamma(\alpha).\]
This motivates the following definition: 
\[(-L)^{-\alpha} = \frac{1}{\Gamma(\alpha)} \int_0^\infty t^{\alpha - 1} S(t) \dd t.\]
In the case where \(\alpha = 1\), we see that the above definition coincides with \(R_0 = (-L)^{-1}\) 
as expected. Moreover, by observing 
\[(-L)^{-1} = (-L)^{-1 + \alpha} (-L)^{-\alpha}\]
as \((-L)^{-1}\) is injective, it follows that \((-L)^{-\alpha}\) is injective for \(\alpha \in (0, 1]\).
This argument can be extended in the case where \(\alpha > 1\) and we have that \((-L)^{-\alpha}\) is
injective for all \(\alpha > 0\). Consequently, we can define \((-L)^\alpha\) to be the 
inverse of \((-L)^{-\alpha}\) for \(\alpha > 0\) where \(\mathcal{D}((-L)^\alpha) = \mathcal{R}((-L)^{-\alpha})\).
With this, we define the interpolation spaces.

\begin{definition}
  For \(\alpha > 0\), we define the interpolation space 
  \[\mathcal{B}_\alpha = \mathcal{D}((-L)^\alpha) = \mathcal{R}((-L)^{-\alpha})\]
  equipped with the norm \(\|x\|_\alpha = \|(-L)^\alpha x\|\). On the other hand, we define \(\mathcal{B}_{-\alpha}\) 
  to be the completion of \(\mathcal{B}\) under the norm \(\|x\|_{-\alpha} = \|(-L)^{-\alpha} x\|\).
\end{definition}

We have the following useful properties.

\begin{proposition}
  \begin{itemize}
    \item For all \(\alpha \ge \beta\) (regardless of sign), we have \(\mathcal{B}_\alpha \subseteq \mathcal{B}_\beta\).
    \item For all \(\alpha > 0, t > 0\), \(S(t)\mathcal{B} \subseteq \mathcal{B}_\alpha\) and 
      \[\|(-L)^\alpha S(t)\|_{\mathcal{B} \to \mathcal{B}} = \|S(t)\|_{\mathcal{B} \to \mathcal{B}_\alpha} \le \frac{C_\alpha}{t^\alpha}.\]
      For this, first consider integer \(\alpha\) and use the ``identity'' 
        \[\frac{1}{2\pi i} \int_{\gamma_{\phi, b}}e^{tz} R_z \dd z = \int_{\gamma_{\phi, b}}\frac{e^{tz}}{z - L} \dd z
          = e^{tL} = S(t).\]
    \item For all \(t \in (0, 1], \alpha \in (0, 1)\) and \(x \in \mathcal{B}_\alpha\), 
      \[\|S(t)x - x\| \le C_\alpha t^\alpha \|x\|_\alpha.\]
  \end{itemize}
\end{proposition}

\end{document}