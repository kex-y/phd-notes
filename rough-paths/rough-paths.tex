\documentclass[]{article}
\usepackage{amsmath}
\usepackage[utf8]{inputenc}
\usepackage[bitstream-charter]{mathdesign}
\usepackage[T1]{fontenc}
\usepackage[dvipsnames]{xcolor}
\usepackage{hyperref}
\hypersetup{
  pdftitle={Rough paths notes},
  pdfauthor={Kexing Ying},
  colorlinks=true,
  linkcolor=Maroon,
  filecolor=Maroon,
  citecolor=Maroon,
  urlcolor=Maroon,
  pdfcreator={LaTeX via pandoc}}
\usepackage{xcolor}
\usepackage[margin = 1.5in]{geometry}
\usepackage{graphicx}
\usepackage{tikz}
\usepackage{tikz-cd}
\usepackage{physics}
\usepackage{amsthm}
\usepackage{titling}
\usepackage{mathtools}
\usepackage{authblk}
\usepackage{todonotes}
\setlength{\parindent}{0pt}
\theoremstyle{definition}
\newtheorem{theorem}{Theorem}
\newtheorem*{theorem*}{Theorem}
\newtheorem{prop}{Proposition}
\newtheorem{corollary}{Corollary}[theorem]
\newtheorem*{remark}{Remark}
\theoremstyle{definition}
\newtheorem{definition}{Definition}
\newtheorem{lemma}{Lemma}[section]
\newtheorem*{lemma*}{Lemma}
\newtheorem{proposition}{Proposition}[section]
\newtheorem*{proposition*}{Proposition}
\newtheorem{example}{Example}[section]

\newcommand{\vertiii}[1]{{\left\vert\kern-0.25ex\left\vert\kern-0.25ex\left\vert #1 
    \right\vert\kern-0.25ex\right\vert\kern-0.25ex\right\vert}}

\let\phi\varphi

\setlength{\parskip}{3pt}

\begin{document}
\title{Rough paths notes}
\author{Kexing Ying}
\maketitle

This note is based on \textit{A course on rough paths} by Peter Friz and Martin Hairer. 

\subsection*{Space of rough paths}

Let \(V\) be a Banach space and \(T > 0\). A \(V\)-valued \(\alpha\)-H\"older rough path is a pair of functions
\[X : [0, T] \to V \text{ and } \mathbb{X} : [0, T]^2 \to V^{\otimes 2}\]
such that 
\[\|X\|_\alpha = \sup_{s \neq t \in [0, T]}\frac{\|X_{s, t}\|}{|s - t|^\alpha}, 
  \|\mathbb{X}\|_{2\alpha} = \sup_{s \neq t \in [0, T]} \frac{\|\mathbb{X}_{s, t}\|}{|s - t|^{2\alpha}} < \infty\]
and moreover, satisfy \textit{Chen's relation}:
\begin{equation}\label{eq:chen}
  \mathbb{X}_{s, t} - \mathbb{X}_{s, u} - \mathbb{X}_{u, t} = X_{s, u} \otimes X_{u, t}
\end{equation}
where we denote \(X_{s, t} = X_t - X_s\).

Recall that the tensor norm is defined as 
\[\|x\| = \inf \left\{\sum \|v\|\|w\| : x = \sum v \otimes w\right\}.\]

Intuitively, the lift \(\mathbb{X}_{s, t}\) is representing \(\int_s^t X_{s, r} \otimes \dd X_r\) and 
Chen's relation follows by asserting that 
\begin{itemize}
  \item the map \(f : \mapsto \int f_r \dd X_r\) is linear;
  \item \(\int_s^t \dd X_r = X_t - X_s\) and
  \item \(\int_s^t f_r \dd X_r = \int_s^u f_r \dd X_r + \int_u^t f_r \dd X_r\).
\end{itemize}

We remark that the lift \(\mathbb{X}\) is not unique from Chen's relation. Indeed, if \(\mathbb{X}\) is a lift, 
then so is \(\mathbb{X}_{s, t} + F_t - F_s\). In fact, they are all of this form. 

We say a rough path \((X, \mathbb{X})\) is weakly geometric if 
\[\text{Sym}(\mathbb{X}_{s, t}) = \frac{1}{2}X_{s, t} \otimes X_{s, t}.\]
This relation is determined by integration by parts.

We introduce the following vocabulary:
\begin{itemize}
  \item the map \(X \mapsto (X, \mathbb{X})\) is called a rough path lift.
  \item \(\mathcal{C}^\infty\) denotes the space of smooth rough paths.
  \item \(\mathcal{L}(\mathcal{C}^\infty) \subseteq \mathcal{C}^\infty\) denotes the canonical lift of a smooth paths.
  \item \(\mathcal{C}^\alpha_g\) the space of weakly geometric rough paths.
  \item \(\mathcal{C}^{0, \alpha}_g = \overline{\mathcal{L}(\mathcal{C}^\infty)}^{\mathcal{C}^\alpha}\).
\end{itemize}

For \(\alpha < \beta\), we have the inclusion \(\mathcal{C}^\beta_g \subseteq \mathcal{C}^{0, \alpha}_g \subseteq \mathcal{C}^\alpha_g\).

We would like to introduce a topology on the space of rough paths via a norm which makes the space of rough paths 
a linear subspace of \(C^\alpha \oplus C_2^{2\alpha}\). By observing that the norm \(\|X\|_{\alpha} + \|\mathbb{X}\|_{2\alpha}\)
does not respect Chen's relation, we remark that this norm does not make the space of rough paths a linear subspace. 
Instead, define 
\[\vertiii{(X, \mathbb{X})}_\alpha = \|X\|_\alpha + \sqrt{\|\mathbb{X}\|_{2\alpha}}.\]
We also record the following (in-homogeneous) \(\alpha\)-H\"older metric on \(\mathcal{C}^\alpha([0, T], V)\):
\[\rho_\alpha(\mathcal{X}, \mathcal{Y}) = \sup_{s \neq t \in [0, T]}\frac{\|X_{s, t} - Y_{s, t}\|}{|s - t|^\alpha} + 
  \sup_{s \neq t \in [0, T]}\frac{\|\mathbb{X}_{s, t} - \mathbb{Y}_{s, t}\|}{|s - t|^{2\alpha}}.\]


The space of rough paths can be alternatively described by a path taking values in a certain additive Lie group. 
Define 
\[T^{(2)}(V) = \mathbb{R} \oplus V \oplus V^{\otimes 2}\]
with the multiplication 
\[(a, b, c) \otimes (a', b', c') = (aa', ab' + a'b, ac + a'c' + b \otimes b').\]
\(T^{(2)}(V)\) is known as the step-2 truncated tensor algebra over \(V\). Then, considering 
\[T^{(2)}_1(V) = \{1\} \oplus V \oplus V^{\otimes 2}\]
with the inherited multiplication from \(T^{(2)}(V)\), we have that \(T^{(2)}_1(V)\) is a Lie group 
with the identity \((1, 0, 0)\) and inverse given by 
\[(1, b, c)^{-1} = (1, -b, b^{\otimes 2} - c).\]
Then, we consider paths taking values in \(T^{(2)}_1(V)\) and in particular, interpret 
\[t \mapsto \mathcal{X}_t = (1, X_t, \mathbb{X}_{0, t}) \in T^{(2)}_1{V}.\]
Chen's relation in this case becomes \(\mathcal{X}_{s, u} \otimes \mathcal{X}_{u, t} = \mathcal{X}_{s, t}\) 
where \(\mathcal{X}_{s, t} = \mathcal{X}_{s}^{-1} \otimes \mathcal{X}_t\).

Denoting \(T^{(2)}_0(V) = V \oplus V^{\otimes 2}\), we define its Lie bracket by 
\[[(b, c), (b', c')] = b \otimes b' - b' \otimes b.\]
This makes \(T^{(2)}_0(V)\) a Lie algebra. Then, denoting \([V, V] = \overline{\langle [v, w] : v, w \in V\rangle}\),
we define \(g^{(2)}(V)\) to be the subalgebra of \(T^{(2)}_0\) defined by 
\[g^{(2)}(V) = V \oplus [V, V].\]
We remark that in \(\mathbb{R}^d\), \([\mathbb{R}^d, \mathbb{R}^d]\) is the space of symmetric matrices 
and so is automatically closed.

Define the map 
\[\exp : T^{(2)}_0(V) \to T^{(2)}_1(V) : (b, c) \mapsto \left(1, b, c + \frac{1}{2}b \otimes b\right).\]
Then, denoting \(G^{(2)}(V) = \exp(g^{(2)}(V)) \le T^{(2)}_1(V)\), it is easy the check that paths 
taking values in \(G^{(2)}(V)\) corresponds to the weakly geometric rough paths.

\subsection*{Integration of one-forms}

Integration of rough paths is motivated by Taylor expansion. Suppose \(F : V \to \mathcal{L}(V, W) \in C^{1 +}\), and 
\(\mathcal{X} = (X, \mathbb{X}) \in \mathcal{C}^\alpha\), then Taylor expansion gives
\[F(X_r) \approx F(X_s) + DF(X_r)X_{s, r}\]
with \(DF(X_t) \in \mathcal{L}(V, \mathcal{L}(V, W)) = \mathcal{L}(V \otimes V, W)\). Then, the Riemann-Stieltjes 
sum can be interpreted as 
\begin{equation}\label{eq:rs-sum}
  \int_s^t F(X_r) \dd \mathcal{X}_r = 
  \lim_{|\mathcal{P}| \to 0} \sum_{[u, v] \in \mathcal{P}} \left(F(X_u)X_{u, v} + DF(X_r)\mathbb{X}_{s, r}\right)
\end{equation}
where we hope the higher order terms vanished. This turns out to be the case for \(\alpha \in (1/3, 1/2]\).

\begin{lemma}\label{lem:control}
  Let \(F : V \to \mathcal{L}(V, W) \in C^2_b\) and \(\mathcal{X} = (X, \mathbb{X})\) be an \(\alpha\)-H\"older 
  rough path with \(\alpha > 1/3\).
  Then, denoting \(Y = F(X)\), \(Y' = DF(X)\) and \(R^Y_{s, t} = Y_{s, t} - Y'_s X_{s, t}\), then 
  \[Y, Y' \in C^\alpha \text{ and } R^Y \in C_2^{2\alpha}.\]
  Moreover, \(\|Y\|_\alpha \le \|DF\|_\infty\|X\|_\alpha, \|Y'\|_\alpha \le \|D^2F\|_\infty\|X\|_\alpha\) 
  and \(\|R^Y\|_{2\alpha} \le \frac{1}{2}\|D^2F\|_\infty\|X\|_\alpha^2\).
\end{lemma}

The above lemma allows us to apply the sewing lemma with increments 
\[\Xi_{s, t} = Y_s X_{s, t} + Y'_s \mathbb{X}_{s, t}\]
resulting in the following theorem.

\begin{theorem}[Lyons]
  Let \(F : V \to \mathcal{L}(V, W) \in C^2_b\) and \((X, \mathbb{X})\) be an \(\alpha\)-H\"older rough path with \(\alpha > 1/3\).
  Then, the limit~\eqref{eq:rs-sum} exists and we have the bound
  \[\left\|\int_s^t F(X_r) \dd \mathcal{X}_r - F(X_s) X_{s, t} - DF(X_s) \mathbb{X}_{s, t}\right\|
    \lesssim_{\alpha}\|F\|_{C^2_b}(\|X\|^3_\alpha + \|X\|_\alpha\|\mathbb{X}\|_{2\alpha})|t - s|^{3\alpha}.\]
\end{theorem}

Since fundamentally, the above theorem relies solely on Lemma~\ref{lem:control}, it is natural to 
make the following definition for the space of integrands.

\begin{definition}
  Let \(X \in C^\alpha([0, T], V)\) and \(Y \in C^\alpha([0, T], \bar W)\). We say \(Y\) is controlled by \(X\) 
  if there exists some \(Y' \in C^\alpha([0, T], \mathcal{L}(V, \bar W))\) and \(R^Y \in C^{2\alpha}_2\) with 
  \(R^Y_{s, t} = Y_{s, t} - Y'_s X_{s, t}\). We write the space of all such pairs of \((Y, Y')\) as 
  \(\mathcal{D}^{2\alpha}_X([0, T], \bar W)\).
\end{definition}

We endow \(\mathcal{D}^{2\alpha}_X([0, T], \bar W)\) with the seminorm 
\[\|Y, Y'\|_{X, 2\alpha} = \|Y'\|_\alpha + \|R^Y\|_{2\alpha}.\]

\begin{theorem}[Gubinelli]
  Let \(\mathcal{X} = (X, \mathbb{X}) \in \mathcal{C}^\alpha([0. T], V)\) and 
  \((Y, Y') \in \mathcal{D}^{2\alpha}_X([0, T], \mathcal{L}(V, W))\) where \(\alpha \in (1/3, 1/2]\).
  Then, the limit 
  \[\int_s^t Y_r \dd \mathcal{X}_r = \lim_{|\mathcal{P}| \to 0} \sum_{[u, v] \in \mathcal{P}}\left(Y_u X_{u, v} + Y'_u \mathbb{X}_{u, v}\right)\]
  exists. Moreover, we have the bound 
  \[\left\| \int_s^t Y_r \dd \mathcal{X}_r - Y_s X_{s, t} - Y'_s \mathbb{X}_{s, t}\right\| 
    \lesssim_{\alpha} (\|X\|_{\alpha}\|R^Y\|_{2\alpha} + \|\mathbb{X}\|_{2\alpha}\|Y'\|_\alpha)|t - s|^{3\alpha}.\]
  Finally, the map 
  \[\Phi : \mathcal{D}^{2\alpha}_X([0, T], \mathcal{L}(V, W)) \to \mathcal{D}^{2\alpha}_X([0, T], W): 
    (Y, Y') \mapsto \left(\int_0^{\cdot} Y_r \dd \mathcal{X}_r, Y\right)\]
  is a continuous linear map between Banach spaces with the bound 
  \[\|\Phi(Y, Y')\|_{X, 2\alpha} \le \|Y\|_\alpha + \|Y'\|_\infty\|\mathbb{X}\|_{2\alpha} + 
    C_\alpha T^\alpha(\|X\|_\alpha \|R^Y\|_{2\alpha} + \|\mathbb{X}\|_{2\alpha}\|Y'\|_{\alpha}).\] 
\end{theorem}

Finally, if \((X, \mathbb{X}) \in \mathcal{C}^\alpha([0, T], V)\) and 
\((Y, Y') \in \mathcal{D}^{2\alpha}_X([0, T], \mathcal{L}(\bar V, bar W))\),
\((Z, Z') \in \mathcal{D}^{2\alpha}_Y([0, T], \bar V)\), then defining 
\[\Xi_{s, t} = Y_s Z_{s, t} + Y'_s Z'_s X_{s, t}\]
where we interpret \(Y'_s \in \mathcal{L}(V, \mathcal{L}(\bar V, \bar W)) \simeq \mathcal{L}(V \otimes \bar V, \bar W)\), 
we may apply the sewing lemma in order to make sense of 
\(\int_s^t Y_r \dd Z_r\).

\subsection*{Stability}

Given \((X, \mathbb{X}), (\tilde X, \tilde{\mathbb{X}}) \in \mathcal{C}^\alpha\) and \((Y, Y') \in \mathcal{D}^{2\alpha}_X\),
\((\tilde Y, \tilde Y') \in \mathcal{D}^{2\alpha}_{\tilde X}\), we define the following notion of ``distance'':
\[\|Y, Y'; \tilde Y, \tilde Y'\|_{X, \tilde X, 2\alpha} = \|Y' - Y\|_{2\alpha} + \|R^Y - R^{\tilde Y}\|_{2 \alpha}.\]
While obviously, this is not a metric as \((Y, Y')\) and \((\tilde Y, \tilde Y')\) are not necessarily in the same space, 
even in the case where \(X = \tilde X\), this distance still does not differentiate between \(Y\) and \(Y + c\). 
Nonetheless, by considering the canonical embedding 
\[\iota_X : \mathcal{D}^{2\alpha}_X \to C^\alpha \oplus C^{2\alpha} : (Y, Y') \mapsto (Y', R^Y),\]
which is an injection if we fix \(Y_0 = \xi\) (since then \(Y_t = \xi + R^Y_{0, t} + Y'_0 X_{0, t}\)), we have that 
\[\|Y, Y'; \tilde Y, \tilde Y'\|_{X, \tilde X, 2\alpha} = \|\iota_X(Y, Y') - \iota_{\tilde X}(\tilde Y, \tilde Y')\|_{\alpha, 2\alpha}.\]
To this end, we have a bound of the form 
\[\|\Phi(Y, Y'); \Phi(\tilde Y, \tilde Y')\|_{X, \tilde X, 2\alpha} \le 
  C(\rho_\alpha(\mathcal{X}, \tilde{\mathcal{X}}) + |Y'_0 - \tilde{Y}'_0| + T^\alpha \|Y, Y'; \tilde Y, \tilde Y'\|_{X, \tilde X, 2\alpha})\]



\end{document}
