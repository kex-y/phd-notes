\documentclass[]{article}
\usepackage{amssymb,amsmath}
\usepackage[utf8]{inputenc}
\usepackage[bitstream-charter]{mathdesign}
\usepackage[T1]{fontenc}
\usepackage[dvipsnames]{xcolor}
\usepackage{hyperref}
\hypersetup{
  pdftitle={Tensor product notes},
  pdfauthor={Kexing Ying},
  colorlinks=true,
  linkcolor=Maroon,
  filecolor=Maroon,
  citecolor=Maroon,
  urlcolor=Maroon}
\usepackage{xcolor}
\usepackage[margin = 1.5in]{geometry}
\usepackage{graphicx}
\usepackage{tikz}
\usepackage{tikz-cd}
\usepackage{physics}
\usepackage{amsthm}
\usepackage{titling}
\usepackage{mathtools}
\usepackage{authblk}
\usepackage{todonotes}
\setlength{\parindent}{0pt}
\theoremstyle{definition}
\newtheorem{theorem}{Theorem}
\newtheorem*{theorem*}{Theorem}
\newtheorem{prop}{Proposition}
\newtheorem{corollary}{Corollary}[theorem]
\newtheorem*{remark}{Remark}
\theoremstyle{definition}
\newtheorem{definition}{Definition}
\newtheorem{lemma}{Lemma}[section]
\newtheorem*{lemma*}{Lemma}
\newtheorem{proposition}{Proposition}[section]
\newtheorem*{proposition*}{Proposition}
\newtheorem{example}{Example}[section]

\let\phi\varphi

\setlength{\parskip}{3pt}

\begin{document}
\title{Tensor product notes}
\author{Kexing Ying}
\date{July 24, 2021}
\maketitle

This document is dedicated to recalling some definitions and theorems regarding 
tensor product of vector spaces. This note is copied directly from my notes for 
my undergrad group representation theory course.

\subsubsection*{Tensor Product of Vector Spaces}

In essence, similar to direct sums, the tensor product of two vector spaces 
in some sense combine them into a larger vector space. In particular, 
while (external) direct sums ``glue'' the vectors of the two spaces, the tensor 
product combined the coefficients of each vectors of the two spaces pairwise.

\begin{definition}[Free Vector Space]
  Let \(S\) be a set and \(\mathbb{F}\) be a field. Then the 
  \(\mathbb{F}\)-free vector space over \(S\) is the set given by 
  \[\mathbb{F}[S] := \left\{ \sum_{s \in S} a_s s \mid 
    |\text{supp}(a)| < \infty \right\}.\]
\end{definition}
We note that we have already seen this definition in the context of monoid 
algebras. The free vector space is equipped with the natural notion of 
addition and scalar multiplication and forms a vector space over \(\mathbb{F}\).

Again, \(s \in S\) corresponds to the elements of the free vector space 
with coefficients \(a_s = 1\) and \(a_r = 0\) for all \(r \neq s\), and in that 
sense, \(S\) is a basis of \(\mathbb{F}[S]\). Thus, with this definition, 
we may consider the free vector space of \(V \times W\). It is clear that 
\(\mathbb{F}[V \times W]\) has a basis of \(V \times W\), and so 
\(\mathbb{F}[V \times W]\) is huge in the sense that
\[\dim \mathbb{F}[V \times W] = |V \times W|.\]
In some sense, the construction of the free vector space forgets the vector 
space structure of \(V\) and \(W\). In particular, we note that for \(v_1, v_2 \in V, w \in W\), 
\((v_1, w) +_{\mathbb{F}[V \times W]} (v_2, w)\) corresponds to the function 
which maps \((v_1, w)\) and \((v_2, w)\) to 1 and 0 otherwise, and thus, 
\((v_1, w) +_{\mathbb{F}[V \times W]} (v_2, w) \neq (v_1 + v_2, w)\). 

This, is however not what we want and so we will quotient it by some subspace.

\begin{definition}[Tensor Product]
  Let \(V, W\) be vector spaces over the field \(\mathbb{F}\) and define 
  \(B(V \times W)\) a subspace of \(\mathbb{F}[V \times W]\) by 
  \[\begin{split}
    B(V \times W) = \text{span}\{& (v_1 + v_2, w) - (v_1, w) - (v_2, w); \\
      & (v, w_1 + w_2) - (v, w_1) - (v, w_2); \\
      & (\lambda v, w) - \lambda(v, w); \\
      & (v, \lambda w) - \lambda(v, w) \mid 
      v, v_1, v_2 \in V, w, w_1, w_2 \in W, \lambda \in \mathbb{F}\}.
  \end{split}\]
  Then the tensor product \(V \otimes W\) is simply 
  \[V \otimes W = \mathbb{F}[V \times W] / B(V \times W).\]
\end{definition}

We denote the image of \((v, w) \in \mathbb{F}[V \times W]\) under the quotient 
map \(\mathbb{F}[V \times W] \to V \otimes W\) by \(v \otimes w\).

\begin{proposition}
  For \(V, W\) vector spaces over the field \(\mathbb{F}\),
  for all \(v, v_1, v_2 \in V, w, w_1, w_2 \in W, \lambda \in \mathbb{F}\), 
  we have 
  \begin{itemize}
    \item \((v_1 + v_2) \otimes w = v_1 \otimes w + v_2 \otimes w\); 
    \item \(v \otimes (w_1 + w_2) = v \otimes w_1 + v \otimes w_2\); 
    \item \((\lambda v) \otimes w = \lambda(v \otimes w)\); 
    \item \(v \otimes (\lambda w) - \lambda(v \otimes w)\).
  \end{itemize}
\end{proposition}
\begin{proof}
  By definition of the quotient.
\end{proof}

\begin{theorem}[Universal Property]
  Let \(V, W, Z\) be vector spaces, then there exists a bilinear map 
  \(F : V \times W \to V \otimes W\) such that for all bilinear maps 
  \(\alpha : V \times W \to Z\), there exists a unique linear map 
  \(\tilde \alpha : V \otimes W \to Z\) such that 
  \(\alpha = \tilde \alpha \circ F\).
\end{theorem}
\begin{proof}
  Let \(\alpha\) be a bilinear map, and we can define \(\bar \alpha\) as the 
  linear extension of \(\alpha\) such that 
  \[\bar \alpha : \mathbb{F}[V \times W] \to Z : 
  \sum_{i = 1}^n a_i (v_i, w_i) \mapsto \sum_{i = 1}^n a_i \alpha(v_i, w_i).\]
  Since \(\alpha\) is bilinear, we have \(B(V \times W) \subseteq \ker \bar \alpha\).
  Thus, by the universal property of the quotient, there exists a unique  
  \(\tilde \alpha : V \otimes W \to Z\) such that 
  \(\tilde \alpha \circ q = \bar \alpha\). 
\end{proof}

One might imagine the universal property as the following 
commutating diagram.
\[\begin{tikzcd}
  V \times W \arrow{r}{\alpha} \arrow[swap]{d}{F} & Z \\
  V \otimes W \arrow[swap]{ur}{\tilde \alpha}&
  \end{tikzcd}\]
In paricular, we have the diagram 
\[\begin{tikzcd}
  V \times W \arrow[hookrightarrow]{r}{\iota} \arrow[swap]{rd}{\alpha}
    & \mathbb{F}[V \times W] \arrow{r}{q} \arrow{d}{\bar \alpha} 
    & V \otimes W \arrow{ld}{\tilde \alpha}\\
  & Z &
\end{tikzcd}\]
and so we may simply \(F = q \circ \iota\) which is bilinear.

Thus, with the universal property, we have obtained a bijection between 
the set of bilinear maps from \(V \times W \to Z\) and the set of linear 
maps between \(V \otimes W \to Z\).

\begin{proposition}
  The universal property determines \(V \otimes W\) uniquely up to isomorphism, 
  i.e. if \(Z\) is a vector space satisfying the universal property of 
  \(V \otimes W\), then \(Z \cong V \otimes W\).
\end{proposition}
\begin{proof}
  Exercise.
\end{proof}

\begin{proposition}
  Let \((v_i)_{i \in I}\) be a basis of \(V\), and \((w_j)_{j \in J}\) be a basis 
  of \(W\). Then, \((v_i \otimes w_j)_{i \in I, j \in J}\) is a basis of 
  \(V \otimes W\).
\end{proposition}
\begin{proof}
  Exercise (Hint: to show linear independence, consider the bilinear map 
  \(B_{nk} : V \times W \to \mathbb{F} : 
  \left(\sum \lambda_i v_i, \sum \mu_j w_j\right) \mapsto \lambda_n \mu_k\) 
  for all \(n, k\)). 
\end{proof}

\begin{proposition}
  There is a canonical linear injection 
  \[V^* \otimes W \to \text{Hom}(V, W),\]
  and this is an isomorphism if \(V\) is finite dimensional.
\end{proposition}
\begin{proof}
  Define 
  \[T : V^* \times W \to \text{Hom}(V, W) : (f, w) \mapsto (v \mapsto f(v)w).\]
  It is easy to see that \(T\) is bilinear and so, by the universal property, 
  there exists a unique linear map \(\tilde T : V^* \otimes W \to \text{Hom}(V, W)\) 
  such that \(\tilde T(f \otimes w) = T(f, w)\). Now, if 
  \(f \otimes w \in \ker \tilde T\), we have either \(f = 0\) or \(w = 0\), and 
  so \(f \otimes w = 0\). Thus, \(\ker \tilde T = \{0\}\) and hence, it is an 
  injection.

  In the case that \(V\) is finite dimensional, we have 
  \[\dim (V^* \otimes W) = \dim V^* \dim W = \dim V \dim W = \dim(\text{Hom}(V, W)),\]
  and so \(T\) is an isomorphism by a dimensional argument.
\end{proof}

\begin{corollary}
  For \(V\) finite dimensional, we have
  \[V \otimes W \cong V^{**} \otimes W \cong \text{Hom}(V^*, W).\]
\end{corollary}

\begin{proposition}
  \(\mathbb{F} \otimes W \cong W\).
\end{proposition}
\begin{proof}
  Exercise.
\end{proof}

\begin{lemma}\label{tensor_map}
  Given linear maps \(T_1 : V \to U_1, T_2 : W \to U_2\), we have 
  \[T_1 \otimes T_2 : V \otimes W \to U_1 \otimes U_2 : 
    v \otimes w \mapsto T_1(v) \otimes T_2(w),\]
  is a linear map. Furthermore, if \(T_i, S_i\) for \(i = 1, 2\) 
  are linear maps on appropriate domains and codomains, 
  \[(T_1 \otimes T_2) \circ (S_1 \otimes S_2) = 
    (T_1 \circ S_1) \otimes (T_2 \circ S_2).\]
\end{lemma}
\begin{proof}
  Easy checks using the universal property.
\end{proof}

\end{document}
